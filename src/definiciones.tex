\chapter{Definiciones cuánticas}\label{ch:definiciones-cuánticas}

\section{Definiciones básicas}\label{sec:definiciones-basicas}

\subsection{Cúbit $\ket{+}$ y $\ket{-}$}\label{subsec:cúbit-$ket{+}$-y-$ket{-}$}

\begin{definition}[Cúbit $\ket{+}$ y $\ket{-}$]
	Definimos el ket $\ket{+}=\frac{\ketC+\ketU}{\sqrt{2}}$ y el ket $\ket{-}=\frac{\ketC-\ketU}{\sqrt{2}}$.
\end{definition}

\begin{exercise}
	Demostrar que $\set{\ket{+}, \ket{-}}$ es una base de $\E$.
	Llamamos \define{base hadamard}{Base Hadamard} a esta base.
\end{exercise}

La matriz cambio de base de la canónica a la direccional es $\frac{1}{2}\mqty(1 & 1 \\ 1 & -1)$, esta matriz es hermitiana y unitaria, por lo tanto esta matriz es también la matriz cambio de base de la direccional a la canónica.

\subsection{Operadores de Pauli}\label{subsec:operadores-de-pauli}

Los operadores de Pauli son 4 aplicaciones del espacio de estados del cúbit que vienen definido sobre la base canónica como:
\begin{table}[htbp]
	\caption{Operadores de Pauli\label{tab:operadores-de-pauli}}
	\centering
	\begin{tabular}{ccccccc}
		\toprule
		Nombre   & Alias         & Valor sobre $\ketC$ & Valor sobre $\ketU$ & Matriz asociada     & Producto exterior               \\
		\midrule
		$\sigma_0$ & I, $\sigma_I$ & $\ketC$             & $\ketU$             & $\mqty({\pmat{0}})$ & $\ketbra{0}{0}+\ketbra{1}{1}$   \\
		$\sigma_1$ & X, $\sigma_x$ & $\ketU$             & $\ketC$             & $\mqty({\pmat{1}})$ & $\ketbra{0}{1}+\ketbra{1}{0}$   \\
		$\sigma_2$ & Y, $\sigma_y$ & $i\ketU$            & $-i\ketC$           & $\mqty({\pmat{2}})$ & $i\ketbra{0}{1}-i\ketbra{1}{0}$ \\
		$\sigma_3$ & Z, $\sigma_z$ & $\ketC$             & $-\ketU$            & $\mqty({\pmat{3}})$ & $\ketbra{0}{0}-\ketbra{1}{1}$   \\
		\bottomrule
	\end{tabular}
\end{table}
