\chapter{Axioma cuánticos}\label{ch:axioma-cuánticos}

En esta sección, vamos a enunciar los axiomas de la mecánica cuántica, los cuales no son resultado de ningún proceso deductivo logico-matemático de conceptos previos.
Son las reglas a las que hemos llegado tras años de observación y experimentación.

\section{Axioma 1. Espacio de estados}\label{sec:axioma-1.-espacio-de-estados}
\begin{definition}[Axioma 1]
	Dado un sistema cuántico, el conjunto de todos los valores posibles se llama el \define{espacio de estados}{Espacio de estados} $\E$, y tiene estructura de $\C$ espacio de Hilbert.
\end{definition}

En computación cuántica, mientras no se diga lo contrario, el espacio de estados tiene dimensión finita.

El estado del sistema viene representado por un vector de $\E$ unitario, donde vectores proporcionales representan el mismo estado.

Usando la notación de Dirac, un elemento del espacio de estados, se llama \define{ket}{Ket} y lo denotamos por $\ket{v}$.

El conjugado de un número complejo $\alpha$, lo denotaremos por $\alpha^*$, y llamaremos conjugado de un ket, al ket cuyas componentes son las componentes conjugadas.
El vector traspuesto de un ket $\ket{v}$ lo denotamos por $\transpose{̣\ket{v}}$.

Si no se dice lo contrario, toda base sobre $\E$ es ortonormal y la denotaremos normalmente por $\set{\ket{i}}$.

\subsection{Producto interno}\label{subsec:producto-interno}
Definimos un producto interno del espacio de estados como una aplicación de $\E\times\E$ en $\C$ que cumple:
\begin{equation*}
	\begin{split}
		\braket{v}{w} & =\braket{w}{v}^* \\
		\braket{\alpha v}{\beta w} & =\alpha^*\beta\braket{v}{w} \\
		\braket{v+v^\prime}{w+w^\prime} &=\braket{v}{w}+\braket{v^\prime}{w}+\braket{v}{w^\prime}+\braket{v^\prime}{w^\prime}
	\end{split}
\end{equation*}

Para una base $\set{\ket{i}}$, todo ket $\ket{v}$ es expresado como combinación lineal $\ket{v}=\sum_{i} v_i\ket{i}$, para ciertos valores $v_i\in\C$, como la base es ortonormal entonces se cumple que $v_j=\braket{j}{v}$, por lo que normalmente expresaremos un ket con respecto a dicha base como:

\begin{equation}
	\label{eq:producto-interno-dirac}
	\ket{v}=\sum_i\braket{i}{v}\ket{i}
\end{equation}

\subsection{El espacio dual}\label{subsec:el-espacio-dual}
El \define{espacio dual}{Espacio dual}, $\E^*$ es el conjunto de todas las aplicaciones lineales de $\E$ en $\C$.
Usando la notación de Dirac, un elemento del espacio dual, se llama \define{bra}{Bra} y lo denotamos por $\bra{v}$.
Todo ket $\ket{v}$, define un bra $\bra{v}$ tan solo con aplicar el producto interno, es decir $\bra{v}(\ket{w})=\braket{v}{w}$.

Si $\set{\ket{i}}$ es una base ortonormal del espacio de estados, tenemos que $\set{\bra{i}}$ es una base ortonormal del espacio dual.

Para cualquier ket $\ket{v}$ sobre la base ortonormal $\set{\ket{i}}$, el bra asociado se expresa como $\bra{v}=\sum_{i}\alpha_i\bra{i}$.
Si calculamos la imagen de $\bra{v}$ sobre la base tenemos que:
\begin{equation*}
	\bra{v}(\ket{i})=\braket{v}{i}\by{\ref{eq:producto-interno-dirac}}\sum_{j}\braket{j}{v}^*\braket{j}{i} = \braket{i}{v}^*
\end{equation*}
Por otra parte tenemos que
\begin{equation*}
	\bra{v}(\ket{i})=\sum_{j}\alpha_j\bra{j}(\ket{i}) = \sum_{j}\alpha_j\braket{j}{i} = \alpha_i
\end{equation*}

Por lo tanto tenemos que
\begin{equation}
	\label{eq:dual_dirac}
	\bra{v}=\sum_i\braket{i}{v}^*\bra{i}
\end{equation}

\section{Axioma 2. Operadores autoadjuntos}\label{sec:axioma-2.-operadores-autoadjuntos}
\begin{definition}[Axioma 2]
	Todas las funciones reales $f$, que caracterizan un sistema físico clásico, tiene asociada en mecánica cuántica un operador autoadjunto $\hat{f}$ que representa el mismo concepto.
\end{definition}
Así por ejemplo, los equivalentes cuánticos a la posición $x$ o el momento $p$ en mecánica clásica, vienen representados por el operador posición $\hat{x}$ o el operador momento $\hat{p}$.

Llamamos observables a los operadores.

\subsection{Producto externo}\label{subsec:producto-externo}
Sea $\ket{v}$ un ket y $\bra{w}$ un bra, se define el producto exterior $\ketbra{v}{w}$ como la aplicación lineal de espacios vectoriales definida por
\begin{equation}
	\label{eq:producto-externo}
	\begin{split}
		\mapsdef{\ketbra{v}{w}}{\E}{\E}{\ket{z}}{\braket{w}{z}\ket{v}}
	\end{split}
\end{equation}

En coordenadas respecto de una base $\set{\ket{i}}$, tenemos que:
\begin{equation*}
	\begin{split}
		\ketbra{v}{w}(\ket{z}) &\by{\ref{eq:producto-interno-dirac}} \sum_{i}\braket{i}{z}\ketbra{v}{w}(\ket{i}\by{\ref{eq:producto-externo}}\sum_{i}\braket{i}{z}\braket{w}{i}\ket{v}\by{\ref{eq:producto-interno-dirac}}\sum_{i}\sum_{j}\braket{i}{z}\braket{w}{i}\braket{j}{v}\ket{j}= \\
		&=\sum_{i}\sum_{j}\braket{w}{i}\braket{j}{v}\braket{i}{z}\bra{j}\by{\ref{eq:producto-externo}}\sum_{i}\sum_{j}\braket{w}{i}\braket{j}{v}\ketbra{j}{i}(\ket{z})\\
	\end{split}
\end{equation*}

Por lo tanto tenemos que:
\begin{equation}
	\label{eq:produto-externo-dirac}
	\ketbra{v}{w} = \sum_{i}\sum_{j}\braket{i}{v}\braket{j}{w}^*\ketbra{i}{j}
\end{equation}

\subsection{Significado del producto externo}\label{subsec:significado-del-producto-externo}

Sea $\set{\ket{i}}$ una base de $\E$, ¿cómo actua $\ketbra{i}{j}$ sobre un ket arbitrario?.

\begin{equation}
	\label{eq:ketbra_base}
	\begin{split}
		\ketbra{i}{j}\left(\ket{v}\right)= & \ketbra{i}{j}\left(\sum_k{\braket{k}{v}\ket{k}}\right)=\sum_k{\braket{k}{v}\ketbra{i}{j}(\ket{k}}) =\\ =&\sum_k{\braket{k}{v}\braket{j}{k}\ket{i}}=\sum_k{\braket{k}{v}\delta_{jk}\ket{i}}=\braket{j}{v}\ket{i}
	\end{split}
\end{equation}

Es decir, $\ketbra{i}{j}$ toma la $j$-ésima coordenada de $v$ y la proyecta sobre la $i$-ésima coordenada, dejando a cero el resto de coordenadas.

Por lo tanto aplicado sobre un ket $\ket{z}$ tenemos
\begin{equation}
	\label{eq:producto-externo-ket-coordenadas}
	\ketbra{v}{w}\left( \ket{z} \right) \by{\ref{eq:produto-externo-dirac}} \sum_{i}\sum_{j}\braket{i}{v}\braket{j}{w}^*\ketbra{i}{j}\left( \ket{z} \right)\by{\ref{eq:ketbra_base}}\sum_{i}\sum_{j}\braket{i}{v}\braket{j}{w}^*\braket{j}{z}\ket{i}
\end{equation}

\subsection{Producto externo como matriz}\label{subsec:producto-externo-como-matriz}
Usando la notación de \eqref{eq:produto-externo-dirac}, podemos considerar la matriz $A$ donde cada entrada es $\ketbramatrix{v}{w}{i}{j}=\braket{i}{v}\braket{j}{w}^*$.

De esta manera, para un ket $\ket{z}$ visto como matriz lo podemos representar por $(\braket{i}{z})_i$, y el producto exterior podemos expresarlo en términos de matrices por:

\begin{equation}
	\label{eq:producto-externo-matriz}
	\ketbra{v}{w}(\ket{z}) = (\sum_{j}\ketbramatrix{v}{w}{i}{j}\braket{j}{z})_i
\end{equation}

\subsection{Definiciones y resultados para observables}\label{subsec:definiciones-y-resultados-para-observables}

\begin{definition}
	Sea $\ketbra{u}{v}$ un observable, definimos la \define{traspuesta hermitiana}{Transpuesta hermitiana} $\ketbra{u}{v}^\dag$ como aquel que tiene como matriz asociada la matriz transpuesta hermitiana de $\ketbra{u}{v}$.
\end{definition}

Como la matriz asociada de $\ketbra{v}{w}$ es $(\ketbramatrix{v}{w}{i}{j})_{ij}$, la matriz traspuesta hermitiana es $(\ketbramatrix{v}{w}{i}{j})_{ij}^\dag = (\braket{i}{v}\braket{j}{w}^*)_{ij}^\dag=(\braket{j}{v}^*\braket{i}{w})_{ij}=(\braket{i}{w}\braket{j}{v}^*)_{ij}=(\ketbramatrix{w}{v}{i}{j})_{ij}$ que es la matriz asociada a $\ketbra{w}{v}$, es decir
\begin{equation}
	\label{eq:traspuesta-hermitiana-producto-externo}
	\ketbra{v}{w}^\dag = \ketbra{w}{v}
\end{equation}

\begin{definition}
	Un observables se dice \define{hermitiano}{Observable hermitiano} si es igual a su traspuesta hermitiana.
\end{definition}

Si $\ketbra{v}{w}$ es hermitiano, tenemos que $\ketbramatrix{v}{w}{i}{j}=\ketbramatrix{v}{w}{j}{i}^*$.
En particular para la diagonal $\ketbramatrix{v}{w}{i}{i}$ la igualada anterior nos dice que es un número real.

\begin{definition}
	Un observable se dice \define{unitario}{Observable unitario} si su traspuesta hermitiana es su inversa.
\end{definition}

\begin{definition}
	Un observable se dice \define{normal}{Observable normal} si conmuta con su traspuesta hermitiana.
\end{definition}

\begin{definition}
	Se llama \define{traza}{Traza de un observable} de un observable a la traza de su matriz asociada.
\end{definition}

Si $\ketbra{v}{w}$ es un observable, tenemos que

\begin{equation}
	\label{eq:traza-como-producto-interno}
	\tr\left(\ketbra{v}{w}\right)=\sum_i \ketbramatrix{v}{w}{i}{i}=\sum_i \braket{i}{v}\braket{i}{w}^*=\braket{v}{w}
\end{equation}

\section{Axioma 4. Colapso de onda}\label{sec:axioma-4.-colapso-de-onda}
\begin{definition}[Axioma 4]
	Dado un estado cuántico $\phi$ de $\E$ y $\hat{f}$ un observable, medir el valor de $\hat{f}$ sobre $\phi$ da como resultado un nuevo estado cuántico $\psi$ tal que $\hat{f}\psi=\lambda\psi$ para algún $\lambda\in\R$.
	El paso del estado $\phi$ a $\psi$ se llama \define{colapso de onda}{Colapso de onda}.
\end{definition}
Así pues, observar un estado cuántico, nos devuelve un autoestado para algún autovalor, pero ¿que significa exactamente esta condición?.

Si $\ketbra{v}{w}$ es un observable, y $\ket{\phi}$ un estado cuántico, y $\psi = \ketbra{v}{w}\left( \ket{\phi} \right)$ entonces debe existir $\lambda\in\R$ tal que
\begin{equation*}
	\begin{align}
		\ketbra{v}{w}\left(\ket{\psi}\right) & = \lambda \ket{\psi} \\
		\sum_{ij}\braket{i}{v}\braket{j}{w}^*\braket{j}{\psi}\ket{i} & = \sum_i \lambda\braket{i}{\psi}\ket{i} && \text{Por \eqref{eq:producto-externo-ket-coordenadas}}\\
		\sum_{j}\braket{i}{v}\braket{j}{w}^*\braket{j}{\psi} & = \lambda\braket{i}{\psi} && \text{Para cada coordenada}
	\end{align}
\end{equation*}

Pero por la propia definición de como se obtiene $\psi$ tenemos que
\begin{equation*}
	\ket{\psi} = \sum_{ij}\braket{i}{v}\braket{j}{w}^*\braket{j}{\phi}\ket{i}\so \braket{i}{\psi} = \sum_{j}\braket{i}{v}\braket{j}{w}^*\braket{j}{\phi}
\end{equation*}

Juntando ambos resultado tenemos que
\begin{equation*}
	\sum_{j}\braket{i}{v}\braket{j}{w}^*\sum_{k}\braket{j}{v}\braket{k}{w}^*\braket{k}{\phi} = \lambda\sum_{j}\braket{i}{v}\braket{j}{w}^*\braket{j}{\phi}

\end{equation*}

\section{Axioma 4. Distribucción de la probabilidad}\label{sec:asioma-4.-distrubicción-de-la-probabilidad}
\begin{definition}[Axioma 4]
	Dado un espacio de estados $\E$, un estado cuántico $\phi$ y un observable $\hat{f}$, la probabilidad de distribucción de la medida observada cumple $E(\hat{f}^n)=\braket{\phi}{\hat{f}^n\phi}$.
\end{definition}
Otra forma de denotar la probabilidad de una medida es $E(\hat{f}^n)=\ev{\hat{f}^n}{\phi}$.

\subsection{Resumen notación cuántica}\label{subsec:resumen-notacion-cuantica}
\begin{table}[htbp]
	\caption{Notación vectorial\label{tab:notacion-vectorial}}
	\centering
	\begin{tabular}{ccccccc}
		\toprule
		Concepto                 & Nomenclatura & Notación        \\
		\midrule
		Vector                     & ket          & $\ket{v}$       \\
		Vector traspuesto          &              & $\transpose{v}$ \\
		Conjugado                  &              & $\alpha^*$      \\
		Vector dual                & bra          & $\bra{v}$       \\
		Norma                      &              & $\norm{v}$      \\
		Producto escalar o interno & braket       & $\braket{v}{w}$ \\
		Producto externo           & ketbra       & $\ketbra{v}{w}$ \\
		\bottomrule
	\end{tabular}
\end{table}

\subsection{Observables}\label{subsec:observables}
Un \define{observable}{Observable} es un operador hermitiano cuyos vectores propios forman una base ortonormal.
Si el espacio de estados es de dimensión finita, todos los operadores hermitianos son observables.

\section{Valores y vectores propios}\label{sec:valores-y-vectores-propios}
En otro momentotyuiooll

\section{Definiciones básicas}\label{sec:definiciones-basicas}

\begin{definition}[Cúbit]
	La unidad mínima de información es el \define{cúbit}{cúbit}, que es un estado cuántico en $\C^2$ con
	dos estados propios.
\end{definition}

El cúbit juega el mismo papel que el bit en la computación clásica.
Los estados propios de un cúbit se denotan por $\ketC$ y $\ketU$ (por analogía a los valores posible de un bit).

\begin{definition}[Base canónica del cúbit]
	Llamaremos base canónica del cúbit a la base formada por sus estados propios.
\end{definition}

\begin{definition}[Base ? del cúbit]
	Llamaremos base ? del cúbit a la formada por los kets $\ket{+}=\frac{\ketC+\ketU}{\sqrt{2}}$ y
	$\ket{-}=\frac{\ketC+\ketU}{\sqrt{2}}$.
\end{definition}

Mientras no se diga lo contrario, todas las bases sobre el espacio de estados se consideran que son ortonormales.

\subsection{Operadores de Pauli}\label{subsec:operadores-de-pauli}

Los operadores de Pauli son 4 aplicaciones del espacio de estados del cúbit que vienen definido sobre la base canónica como:
\begin{table}[htbp]
	\caption{Operadores de Pauli\label{tab:operadores-de-pauli}}
	\centering
	\begin{tabular}{ccccccc}
		\toprule
		Nombre   & Alias & Valor sobre $\ketC$ & Valor sobre $\ketU$ & Matriz asociada     & Producto exterior               \\
		\midrule
		$\sigma_0$ & I     & $\ketC$             & $\ketU$             & $\mqty({\pmat{0}})$ & $\ketbra{0}{0}+\ketbra{1}{1}$   \\
		$\sigma_1$ & X     & $\ketU$             & $\ketC$             & $\mqty({\pmat{1}})$ & $\ketbra{0}{1}+\ketbra{1}{0}$   \\
		$\sigma_2$ & Y     & $\ketU i$           & $-\ketC i$          & $\mqty({\pmat{2}})$ & $i\ketbra{0}{1}-i\ketbra{1}{0}$ \\
		$\sigma_3$ & Z     & $\ketC$             & $-\ketU$            & $\mqty({\pmat{3}})$ & $\ketbra{0}{0}-\ketbra{1}{1}$   \\
		\bottomrule
	\end{tabular}
\end{table}
